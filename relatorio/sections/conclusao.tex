\section{Conclusão}

 Pôde-se notar do gráfico presente na Figura \ref{fig:Fig3M} que os códigos convolucionais são mais eficientes que todos os outros códigos implementados, sendo que, quanto mais memórias a máquina de estados possui, melhor é o desempenho do código.
 
 Quanto às três diferentes métricas adotadas para a implementação do algoritmo de Viterbi, pôde-se inferir do gráfico presente na Figura \ref{fig:Fig2M} que o melhor método é o utilizando a distância Euclidiana. O método da probabilidade é levemente superior ao método da distância de Hamming - o que era esperado, dado que o segundo é uma aproximação do primeiro -, mas ambos perdem consideravelmente para a distância Euclidiana.