\section{Conclusão}

 Pôde-se notar do gráfico presente na Figura \ref{fig:Fig3M} que os códigos convolucionais são mais eficientes que todos os outros códigos implementados, sendo que, quanto mais memórias a máquina de estados possui, melhor é o desempenho do código.
 
 Quanto às três diferentes métricas adotadas para a implementação do algoritmo de Viterbi, pôde-se inferir do gráfico presente na Figura \ref{fig:Fig2M} que o melhor método é o utilizando a distância Euclidiana. O método da probabilidade exata é levemente superior ao método da distância de Hamming - o que era esperado, dado que o segundo é uma aproximação do primeiro - mas ambos perdem consideravelmente para o da distância Euclidiana. A superioridade deste último é devida a representar uma \textit{soft decision}, enquanto os outros dois representam \textit{hard decisions}.

 Por último, nota-se que a maioria dos códigos desenvolvidos ao longo do semestre esteve distante de atingir a capacidade do canal ($\frac{E_i}{N_0}$ diferindo em mais que 3dB, em média). As exceções foram os códigos convolucionais, em particular aqueles que usaram Viterbi com distância Euclidiana para decodificação.