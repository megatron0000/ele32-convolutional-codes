\section{Respostas às perguntas}

\subsection{Perguntas presentes em \cite{ref:roteiro3}}

\subsubsection{Quais foram as maiores dificuldades em implementar o codificador convolucional?}

Não houve dificuldade. Modelando o codificador como uma máquina de estado, a implementação foi direta.

\subsubsection{Quanto tempo a sua solução demora para codificar cada bit de informação? Faça uma média.}

Analisando cada um dos 3 códigos convolucionais pedidos no roteiro, observou-se que o primeiro demora 330ns para codificar cada bit de informação; o segundo, 365ns; o terceiro, 443ns.

\subsubsection{Quais foram as maiores dificuldades em implementar o decodificador convolucional?}

Dado que a decodificação tem complexidade exponencial de espaço na quantidade de "memórias" do código convolucional, o maior entrave à implementação do algoritmo foi vislumbrar uma estrutura de dados não-ingênua que permitisse codificar toda a informação necessária sem uso de memória excessiva (por exemplo, liberando espaço de memória ao descartar possíveis caminhos na treliça em meio ao algoritmo).

\subsubsection{Como a probabilidade de erro de transmissão foi estimada? Qual é o seu valor? Como ela se
compara com o valor de p escolhido? Como ela muda com m?}

A probabilidade de erro de transmissão foi estimada por meio da relação sinal ruído normalizada $(\frac{E_{i}}{N_{0}})$. Seu valor foi obtido pela equação $p(\frac{E_{b}}{N_{0}}) = Q(\sqrt{\frac{2E_{b}}{N_{0}}})$, que é a mesma para os dois canais utilizados nesse roteiro (BSC e AWGN), a qual compara a relação sinal ruído normalizada com seu valor de $p$. Convertendo a energia de bit transmitido $(E_{b})$ para a energia de bit de informação $(E_{i})$, tal que $E_{b} = \frac{E_{i}}{3}$, obtém-se a equação final utilizada $p(\frac{E_{i}}{N_{0}}) = Q(\sqrt{\frac{2E_{i}}{3N_{0}}})$. A partir do gráfico da Figura \ref{fig:Fig2M}, ao fixar-se um valor específico de $p$, obtém-se valores distintos de $(\frac{E_{i}}{N_{0}})$ para cada valor de $m$, de tal forma que, quanto maior for o valor de $m$, menor será o valor de $\frac{E_{i}}{N_{0}}$.  

\subsubsection{Qual é o tamanho final do seu executável?}

O programa foi codificado em linguagem Julia, a qual usa uma técnica de compilação "just in time" para converter o código a instruções de uma LLVM ("low level virtual machine") e o executar. Apesar de existirem desenvolvedores contribuindo para tornar possível a compilação prévia (ou seja, tornar possível a geração de um executável "standalone"), o código deste laboratório não pôde ser convertido num executável puro. Ainda assim, pode-se dizer, sobre os arquivos produzidos na linguagem Julia, que nenhum supera 2Kbytes de tamanho e, juntos, não superam 20Kbytes.

\subsubsection{Quanto tempo a sua solução demora para decodificar cada bit? Faça uma média.}

Os decodificadores de cada código convolucional trabalhado demoram, em média, $30 \mu s$, $60 \mu s$ e $300 \mu s$ para decodificar 1 bit de informação.

\subsection{Perguntas presentes em \cite{ref:roteiro4}}

\subsubsection{Compare todos com a capacidade do canal Gaussiano estabelecendo o valor de $(\frac{E_{b}}{N_{0}})$ que teoricamente permitiria a transmissão com probabilidade de erro tão baixa quanto se queira}

Para obter o valor da razão sinal ruído utiliza-se a equação de Shannon-Hartley.

\begin{equation}
C=B\log_{2}(1+\frac{E_{b}}{N_{0}})
\end{equation}

Convertendo a energia de bit transmitido $(E_{b})$ para a energia de bit de informação $(E_{i})$, tal que $E_{b} = \frac{E_{i}}{3}$, obtém-se a equação final utilizada.

\begin{equation}
C=\frac{1}{2}\log_{2}(1+\frac{E_{i}}{3N_{0}})
\end{equation}

A partir das taxas de cada código - $\frac{4}{7}$ para os códigos de bloco, $\frac{7}{12}$ e $\frac{8}{14}$ para os códigos cíclicos e $\frac{1}{3}$ para os códigos convolucionais - obtém-se o valor de $(\frac{E_{i}}{N_{0}})$.

\begin{table}[H]
	\centering
	\captionsetup{font=scriptsize}
	\captionof{table}{Valores de $(\frac{E_{b}}{N_{0}})$ para probabilidade de erro tão baixa quanto se queira\label{table:snrvalues1}}
	\begin{tabular}{c|c}
		\textbf{Código} & \textbf{$E_{i}/N_{0} (dB)$} \\ \hline
		\textbf{Códigos de bloco} & 5.59 \\
		\textbf{Código cíclico (7,12)} & 5.72 \\
		\textbf{Código cíclico (8,14)} & 5.59 \\
		\textbf{Códigos convolucionais} & 2.46
	\end{tabular}
\end{table}

\subsubsection{Compare os valores de $(\frac{E_{b}}{N_{0}})$ de fato necessário para transmitir com os sistemas acima com probabilidade de erro de $10^{-4}$}

Observando a Figura \ref{fig:Fig3M} obtemos os valores de $(\frac{E_{b}}{N_{0}})$ para probabilidade de erro de $10^{-4}$ traçando uma reta. Os valores obtidos estão representados na Tabela \ref{table:snrvalues2}.

\begin{table}[H]
	\centering
	\captionsetup{font=scriptsize}
	\captionof{table}{Valores de $(\frac{E_{b}}{N_{0}})$ para probabilidade de erro de $10^{-4}$\label{table:snrvalues2}}
	\begin{tabular}{c|c}
	\textbf{Código} & \textbf{$E_{i}/N_{0} (dB)$} \\ \hline
	\textbf{Euclidean (m=6)} & 2.9 \\
	\textbf{Euclidean (m=4)} & 3.9 \\
	\textbf{Euclidean (m=3)} & 4.2 \\
	\textbf{Exact (m=6)} & 5.1 \\
	\textbf{Hamming (m=6)} & 5.2 \\
	\textbf{Hamming (m=4)} & 5.9 \\
	\textbf{Exact (m=4)} & 6 \\
	\textbf{Exact (m=3)} & 6.5 \\
	\textbf{Hamming (m=3)} & 6.6 \\
	\textbf{Bloco Lab1} & 8.6 \\
	\textbf{BCH} & 9 \\
	\textbf{Cyclic (8,14)} & \textgreater{}10 \\
	\textbf{Bloco Hamming} & \textgreater{}10 \\
	\textbf{Cyclic (7,12)} & \textgreater{}10
\end{tabular}
\end{table}